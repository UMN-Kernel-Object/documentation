\documentclass[aspectratio=169]{beamer}
\usetheme{Madrid}
\usepackage{amsfonts}
\usepackage{pifont}
\usepackage{newpxmath}
\usepackage{newpxtext}

\date{September 25, 2024}

\begin{document}

\begin{frame}{Project Subgroup}
	\pause
	\begin{block}{Goal}
		This subgroup's goal is to make a meaningful and useful patchset for the kernel, and get it accepted upstream after passing appropriate review.

		\vspace{1em}

		A secondary goal is to let members newer to kernel programming see how more experienced members solve different problems in the kernel, and to encourage dialogue between these groups.
	\end{block}
	\begin{itemize}
		\item This kind of thing is how we get unbanned from the kernel!
		\item It'll also hopefully turn into an academic publication, which is nice to have on a resume
	\end{itemize}
\end{frame}

\begin{frame}{Organization}
	\begin{itemize}
		\item<1-> Mostly online, through Discord and GitHub
			\begin{itemize}
				\item ``Real'' kernel work doesn't use GitHub, but this will make it much easier for people to dip their toes in
			\end{itemize}
		\item<2-> Three main prongs
			\begin{itemize}
				\item<2-> Development work on the project -- hopefully most people in the subgroup make some change at some point, even if it's ``just'' working on docs, comments, or tests
				\item<3-> Development streams with Q\&A -- a core group of 2-5 volunteers that commit to doing 2-3 hours of dev work every 2 weeks, streaming it on Discord, and answering questions from people who join the call (e.g. ``why are you using \texttt{GFP\_KERNEL\_ACCOUNT} here, what does that flag mean?'')
				\item<4-> Code review with the whole group -- this might be every other or every third meeting in the spring?
			\end{itemize}
	\end{itemize}
\end{frame}

\begin{frame}[fragile]{Project Idea -- Problem}
	\begin{itemize}
		\item This code runs as root, and ensures that files in a project directory can be accessed by everyone in the project's group
	\end{itemize}

	\begin{example}
		\begin{semiverbatim}\input{before.c}\end{semiverbatim}
	\end{example}

	\begin{itemize}
		\pause
		\item This code has a security bug!
		      \pause
		\item What if an attacker replaces \texttt{path} with a hard-link to \texttt{/bin} between \texttt{stat} and \texttt{chown}?
	\end{itemize}
\end{frame}

\begin{frame}{Project Idea -- Solution}
	\begin{itemize}
		\item A solution I thought of:
		      \begin{itemize}
			      \item Have a version of \texttt{stat} that sets a flag on the file (the dirent)
			      \item Any modification of the file clears the flag, no matter who does it
			      \item Have a version of \texttt{chown} that only works if the flag is set -- otherwise, it returns an error
			            \begin{itemize}
				            \item As with all modifications, this version of \texttt{chown} clears the flag when it modifies the file
			            \end{itemize}
		      \end{itemize}
		\item The project is to implement this in the kernel -- also other syscalls than \texttt{stat} and \texttt{chown} (e.g. \texttt{access}, \texttt{chmod}, \texttt{unlink}, ...)
		\item This is inspired by a CPU feature called LL/SC\footnote{\url{https://en.wikipedia.org/wiki/Load-link/store-conditional}}, so I'm calling it ``LL/SC dirops'' as a working name -- I'm open to suggestions for a better one!
		\item Aim to get v1 done this fall, then we get to code review, benchmarking, optimization, etc. in the spring
		      \begin{itemize}
			      \item Hopefully, the version for just \texttt{stat} and \texttt{chown} is done before Thanksgiving
		      \end{itemize}
	\end{itemize}
\end{frame}

\begin{frame}{Project Subgroup -- Get Involved}
	\begin{itemize}
		\item Join the \texttt{\#llsc-dirops} channel on Discord
		      \begin{itemize}
			      \item Use the \texttt{/github-invite} command in that channel to get added to the \texttt{github.com} repo
		      \end{itemize}
		      \vspace{1em}
		\item Are you excited to get your hands dirty doing development work?
		      \begin{itemize}
			      \item Volunteer to run dev streams!
			      \item (Or just hack on the code if streaming sounds stressful)
		      \end{itemize}
		      \vspace{1em}
		\item Are you interested in learning more but not confident in your ability to help?
		      \begin{itemize}
			      \item Join the dev streams and ask questions!
		      \end{itemize}
	\end{itemize}
\end{frame}

\end{document}
