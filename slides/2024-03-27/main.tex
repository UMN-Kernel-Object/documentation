\documentclass[aspectratio=169, notes]{beamer}
% \setbeamertemplate{note page}[plain]
% \setbeameroption{show notes on second screen=right}
\usepackage{amsfonts}
\usepackage{pifont}
\usepackage{newpxmath}
\usepackage{newpxtext}

\usepackage{listings}
\usepackage{tikz}
\usetikzlibrary{quotes}

\definecolor{c1}{RGB}{ 86, 180, 233}
\definecolor{c2}{RGB}{  0, 158, 115}
\definecolor{c3}{RGB}{230, 159,   0}
\definecolor{c4}{RGB}{213,  94,   0}
\definecolor{c5}{RGB}{204, 121, 167}
\definecolor{c6}{RGB}{255, 170,  20}

\lstdefinelanguage{Nix}{
	classoffset=0,
	morekeywords={let,in},
	morecomment=[l]{\#},
	morecomment=[s]{/*}{*/},
	moredelim=*[s][stringstyle]{"}{"},
	moredelim=*[s][\color{c1}]{\$\{}{\}},
	moredelim=*[is][\color{black}]{<|}{|>},
	moredelim=*[is][\color{c4}]{<:}{:>},
	% Highlight literals
	classoffset=1,
	morekeywords={true},
	keywordstyle=\color{c4}\bfseries,
}
\lstset{
	basicstyle=\small\ttfamily,
	commentstyle=\color{gray},
	keywordstyle=\color{c2}\bfseries,
	numberstyle=\color{c3},
	stringstyle=\color{c3},
	showstringspaces=false
}

\newcommand{\todo}[1]{\note[item]{\textcolor{c4}{\textbf{TODO:}} #1}}

\newcommand{\yes}{{\color{c2}\ding{51}}}
\newcommand{\no}{{\color{c4}\ding{55}}}

\title{Nix(OS)}
\date{March 27, 2024}

\begin{document}

\frame{\titlepage}

\begin{frame}{\texttt{\$ whoami}}
	\begin{itemize}
		\item Nathan Ringo (\texttt{remexre}\footnote{\url{https://remexre.com/}})
		\item PhD student at UMN (MELT\footnote{\url{https://melt.cs.umn.edu/}} lab: PL, compilers, attribute grammars)
		\item Researcher at SIFT\footnote{\url{https://sift.net/}} (program analysis, network security, etc.)
		\item Nixpkgs maintainer (only of one package, \texttt{linenoise})
		\item Nix user for $\sim$4 years, NixOS for $\sim$3, Linux for $\sim$15
	\end{itemize}
\end{frame}

\begin{frame}
	\begin{center}
		\begin{tikzpicture}[thick, every edge quotes/.style={fill=white,font=\footnotesize}]
			\draw[color=white] (-5, -3) rectangle (5, 4);
			\draw (0, 0.5) node {\includegraphics[width=1cm]{nix-snowflake.png}};
			\draw ({3.25 * sin(0)  }, {3.25 * cos(0)  }) node {the language};
			\draw ({3.75 * sin(120)}, {3.75 * cos(120)}) node {the operating system};
			\draw ({3.75 * sin(240)}, {3.75 * cos(240)}) node {the build system};
			\draw[color=c2] ({0.5 * sin(0)  }, {0.5 * cos(0)   + 0.5}) edge["is"] ({3 * sin(0)  }, {3 * cos(0)  });
			\draw[color=c2] ({0.5 * sin(120)}, {0.5 * cos(120) + 0.5}) edge["is"] ({3 * sin(120)}, {3 * cos(120)});
			\draw[color=c2] ({0.5 * sin(240)}, {0.5 * cos(240) + 0.5}) edge["is"] ({3 * sin(240)}, {3 * cos(240)});
			\draw[color=c4] ({3 * sin(0)  }, {3 * cos(0)  }) edge["is not"] ({3 * sin(120)}, {3 * cos(120)});
			\draw[color=c4] ({3 * sin(120)}, {3 * cos(120)}) edge["is not"] ({3 * sin(240)}, {3 * cos(240)});
			\draw[color=c4] ({3 * sin(240)}, {3 * cos(240)}) edge["is not"] ({3 * sin(0)  }, {3 * cos(0)  });
		\end{tikzpicture}
	\end{center}
\end{frame}

\begin{frame}{Nix (the language)}
	\begin{itemize}
		\item ``JSON plus functions''
		      \begin{itemize}
			      \item Unfortunately, the syntax is a bit weird
			      \item Dynamically typed, lazily evaluated, purely functional
			      \item \texttt{let} expressions like in OCaml
			      \item Lambdas are \texttt{arg: body}, function calls are \texttt{f x1 x2}
			      \item Lambdas can destructure with \texttt{\{ attr1, attr2 ? default\}: body}
			      \item String interpolation with \texttt{"...\$\{expr\}..."}
		      \end{itemize}
	\end{itemize}
	\lstinputlisting[language=Nix]{nix-code.nix}
\end{frame}

\begin{frame}{Nix (the build system)}
	\begin{itemize}
		\item Stores files in a \emph{content-addressed store}
		      \begin{itemize}
			      \item \texttt{/nix/store/\$\{hash\}-\$\{name\}}
		      \end{itemize}
		\item Builds are sandboxed while they happen
		      \begin{itemize}
			      \item Can only read from the Nix store
			      \item Must declare dependencies
			      \item Only write to their output directory
		      \end{itemize}
		\item \emph{Either}:
		      \begin{itemize}
			      \item Builds cannot access the network, and the \texttt{\$\{hash\}} is the hash of the build script
			      \item Builds can access the network, and \texttt{\$\{hash\}} is the hash of the output
		      \end{itemize}
	\end{itemize}
\end{frame}

\begin{frame}{Derivations}
	\begin{itemize}
		\item Nix (the language) is used to write build scripts for Nix (the build system), which are called \emph{derivations}
	\end{itemize}
	\lstinputlisting[language=Nix,basicstyle=\footnotesize\ttfamily]{derivation.nix}
\end{frame}

\begin{frame}{Nixpkgs and Flakes}
	\begin{itemize}
		\item Nixpkgs is the standard library for Nix (the language), and the main repository for Nix (the build system)
		\item Traditionally imported as \texttt{pkgs}; packages might be e.g. \texttt{pkgs.hello} (GNU Hello), \texttt{pkgs.firefox} (Firefox), \texttt{pkgs.python311Packages.ipython} (IPython)
		\item Important functions: \texttt{pkgs.callPackage}, \texttt{pkgs.stdenv.mkDerivation}
		      \begin{itemize}
			      \item \texttt{pkgs.callPackage} is \texttt{import} but better
			      \item \texttt{pkgs.stdenv.mkDerivation} is \texttt{builtins.derivation} but better
		      \end{itemize}
		      \vspace{1cm}
		\item Flakes are ``a Git repo with Nix code''
		\item Flakes can depend on flakes, pin to versions, etc.
		\item e.g. the Neovim text editor, the OCaml LSP, a utility library for customizing Rust toolchains (fenix)
	\end{itemize}
\end{frame}

\begin{frame}{What nice things does \texttt{stdenv.mkDerivation} give me?}
	\begin{itemize}
		\item Lots of nice default behavior for:
		      \begin{itemize}
			      \item Depending on packages from various ecosystems easily
			      \item Building ``\texttt{./configure \&\& make \&\& make install}'' software easily
			      \item Handling cross-compilation
			      \item Checking license compliance
		      \end{itemize}
		\item Has hooks for:
		      \begin{itemize}
			      \item Patching software (source or binaries) to work better with Nix
			      \item Integrating other build systems, e.g. CMake, Meson, Ninja
		      \end{itemize}
		\item Along with \texttt{pkgs.callPackage}, allows users to customize packages:
		      \begin{itemize}
			      \item \texttt{pkgs.openssl.override \{ withCryptodev = true; \}}
			      \item \texttt{pkgs.openssl.overrideAttrs (old: \{} \\
			            \hspace{1cm}\texttt{patches = old.patches ++ [ ./foo.patch ];} \\
			            \texttt{\})} \\
		      \end{itemize}
	\end{itemize}
\end{frame}

\begin{frame}{The package I maintain}
	\url{https://github.com/NixOS/nixpkgs/blob/master/pkgs/development/libraries/linenoise/default.nix}
\end{frame}

\begin{frame}{Developing with Nix}
	\begin{itemize}
		\item You can define a \texttt{devShell} with \texttt{pkgs.mkShell}
		      \begin{itemize}
			      \item Has dependencies like a normal derivation
			      \item \texttt{inputsFrom} -- derivations to copy dependencies from
			      \item \texttt{shellHook} -- some bash code to run when entering the shell
		      \end{itemize}
		\item You can enter a shell whenever you enter a directory with \href{https://direnv.net/}{direnv}
		      \vspace{1cm}
		\item Each project gets \emph{exactly} the same dependencies on every developer's machine
		\item Developers don't need to globally install those dependencies; each project's dependencies are isolated
		\item Developers can keep using their preferred shell, preferred editor, etc.
	\end{itemize}
\end{frame}

\begin{frame}{NixOS}
	\begin{itemize}
		\item Generate configs for every package on your OS as Nix packages, and a script to deploy them to the current machine
		\item Win \#1: You can reproduce the config on other machines, or in CI
		\item Win \#2: You can roll back to a previous config easily
		\item Win \#3: You can share snippets of config as code
	\end{itemize}
\end{frame}

\begin{frame}{Vocab and Q\&A}
	\begin{center}
		\begin{tabular}{ c | c }
			Nix           & Rest of the world      \\
			\hline\hline
			Derivation    & package (built by Nix) \\
			Flake         & package (of Nix code)  \\
			Attribute     & key-value pair         \\
			Attribute Set & dict, hash, table, map
		\end{tabular}
	\end{center}
	\vspace{1cm}
	\begin{center}
		{\huge Questions?}
	\end{center}
	\vspace{1cm}
	{\hfill Next up: \href{https://github.com/UMN-Kernel-Object/scripts/blob/main/kern-me-up.sh}{\texttt{kern-me-up.sh}}}
\end{frame}

\end{document}
