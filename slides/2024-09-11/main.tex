\documentclass[aspectratio=169, notes]{beamer}
\setbeamertemplate{note page}[plain]
\setbeameroption{show notes on second screen=right}
\usepackage{amsfonts}
\usepackage{pifont}
\usepackage{newpxmath}
\usepackage{newpxtext}

\definecolor{umn maroon}{RGB}{122,   0,  25}
\definecolor{umn gold}{RGB}{255, 204,  51}

\newcommand{\todo}[1]{\note[item]{\textcolor{red}{\textbf{TODO:}} #1}}

\begin{document}

{
\setbeamercolor{background canvas}{bg=umn gold}
\newgeometry{margin=0cm}
\setbeamertemplate{navigation symbols}{}
\setbeamertemplate{background canvas}[default]
\frame{\includegraphics[width=\textwidth]{title-page.pdf}}
}

\begin{frame}{Agenda}
	\begin{itemize}
	\item Pizza and Conversation
		\begin{itemize}
		\item Grab some pizza and introduce yourself to your neighbors
		\item \textbf{Sign in} on the attendance sheet -- this is how we get funding for more pizza!
		\item Check the checkbox if you want to be added to our mailing list
		\end{itemize}
	\item Officer and advisor introduction
	\item Group overview and logistics
	\item Linux kernel
	\item Wrap-up
	\end{itemize}
\end{frame}

\begin{frame}{Officer introductions}
	\begin{itemize}
	\item Devajya Khanna (\texttt{@devajya}), President
		\begin{itemize}
		\footnotesize
		\item CSci junior, distributed systems research
		\item
		\includegraphics[width=1em]{vim.png}
		\includegraphics[width=1em]{logo-ubuntu.png}
		\includegraphics[width=1em]{java.png}
		\end{itemize}
	\item Nathan Ringo (\texttt{@remexre}), Vice President
		\begin{itemize}
		\footnotesize
		\item CSci PhD student (programming languages), researcher at \href{https://sift.net/}{SIFT} (security and systems)
		\item
		\includegraphics[width=1em]{neovim.png} (\includegraphics[width=1em]{vim.png})
		/
		\includegraphics[width=1em]{nix.png}
		/
		\includegraphics[width=1em]{haskell.png}
		\includegraphics[width=1.5em]{lisp.png}
		\includegraphics[width=1.25em]{rust.png}
		\end{itemize}
	\item Hafsa Abdirahman (\texttt{@hafsa}), Secretary
	\item Anirudh Krishnakumar (\texttt{@anirudh29.}), Treasurer
		\begin{itemize}
		\footnotesize
		\item Junior in computer science and business management
		\item
		\includegraphics[width=1em]{grin.png}
		\includegraphics[width=1em]{tennis.png}
		\includegraphics[width=1em]{hamburger.png}
		\end{itemize}
	\item Sanjali Roy (\texttt{@sunjelly}), Communications Officer
		\begin{itemize}
		\footnotesize
		\item Senior undergrad computer science \& psychology student
		\item
		\includegraphics[width=1em]{flag_in.png}
		\includegraphics[width=1em]{guide_dog.png}
		\includegraphics[width=1em]{computer.png}
		\includegraphics[width=1em]{guitar.png}
		\includegraphics[width=1em]{books.png}
		\end{itemize}
	\end{itemize}
	\todo{Make sure everyone fills this out}
\end{frame}

\begin{frame}{Advisor introductions}
	\begin{itemize}
	\item Jack Kolb
		\begin{itemize}
		\item UMN teaching professor and alumnus
		\item Research experience in computing systems
		\item
		\includegraphics[width=1.5em]{vim.png}
		\includegraphics[width=2em]{archlinux.png}
		\includegraphics[width=2em]{python.jpg}
		\includegraphics[width=2em]{C.jpg}
		\end{itemize}
	\item Alex Elder
		\begin{itemize}
		\item UMN alumnus and kernel maintainer / developer
		\item Expertise on development within the Linux kernel community
		\item \includegraphics[width=2em]{C.jpg}
		      \includegraphics[width=1.5em]{tux.png}
		      \includegraphics[width=1.5em]{vim.png}
		      \includegraphics[width=1.5em]{logo-ubuntu.png}
		      \includegraphics[width=1.5em]{cookie.jpg}
		\end{itemize}
	\end{itemize}
	\todo{Make sure everyone fills this out}
\end{frame}

\begin{frame}{Group overview}
	\begin{itemize}
	\item What is UKO?
		\begin{itemize}
		\item UMN Kernel Object (\texttt{umn.ko})
		\item In Linux, kernel modules have the \texttt{.ko} file extension
		\end{itemize}
	\item What is UKO focused on?
		\begin{itemize}
		\item The Linux kernel (and other operating systems, and free software generally)
		\item Discussion of the technical and social aspects about the above
		\end{itemize}
	\item What are UKO's long-term goals?
		\todo{Revise long-term goals to align with mission statement}
		\begin{itemize}
		\item Take on some more significant kernel-related projects
		\item Pursue members' interests and goals
		\item Mend our relationship with the Linux community and restore the University's ability to contribute to Linux
		\end{itemize}
	\end{itemize}
\end{frame}

\begin{frame}{Group logistics}
	\begin{itemize}
	\item How is UKO organized?
		\begin{itemize}
		\item Whole-group meetings every 2 weeks
		\item Subgroup meetings
		\end{itemize}
	\item What are UKO's goals this semester?
		\begin{itemize}
		\item Get everyone hacking on some kernel code
			\begin{itemize}
			\item Become familiar with the Linux contribution process
			\item Learn to work with and extend a large codebase
			\end{itemize}
		\item Build the expertise needed for members to make a more significant kernel-related project
		\end{itemize}
	\end{itemize}
\end{frame}

\begin{frame}{Subgroups}
	\begin{itemize}
	\item If you want to lead a subgroup, just ask!
	\item Tell us:
		\begin{itemize}
		\item What you plan to do in your subgroup
		\item How it ties into the group's goals and the Linux kernel
		\end{itemize}
	\item Next meeting, we'll poll members for what subgroup(s) they're interested in
	\item Already seen interest in a Rust subgroup
		\begin{itemize}
		\item In the past, we've had subgroups that focused on schedulers and xv6
		\end{itemize}
	\item Propose your subgroup by \strong{next Wednesday}, September 18th
	\end{itemize}
\end{frame}

\section{Alex}

\begin{frame}{What is Linux?}
	\pause
	\begin{itemize}
	\item Linux ``started'' August 25, 1991 when Linus Torvalds announced his Minix-like operating system on the ``\texttt{comp.os.minux}'' newsgroup
		\begin{itemize}
		\item ``just a hobby, won't be big and professional like gnu''
		\end{itemize}
	\item Released with a GNU GPL license; probably the most successful ``open source'' project
	\item Gradually attracted contributors in the early 1990's
		\begin{itemize}
		\item Red Hat got its start when a technical guy packaging a ``distribution'' met a business guy with big ideas
		\end{itemize}
	\item Started gaining commercial success in the late '90's
		\begin{itemize}
		\item Red Hat IPO in 1999
		\end{itemize}
	\end{itemize}
\end{frame}

\begin{frame}{Where is Linux used?}
	\begin{itemize}
	\item It runs on an extremely wide range of hardware
		\begin{itemize}
		\item Raspberry Pi, or even ``IoT'' devices
		\item Mobile phones and tablets (Android/Chrome OS)
		\item Laptops and Desktops
		\item Cloud computing platforms
		\item Supercomputers
		\end{itemize}
	\item It is freely available
		\begin{itemize}
		\item Adding value can be worth money though
		\end{itemize}
	\end{itemize}
\end{frame}

\begin{frame}{Open source}
	\begin{itemize}
	\item GNU Public License (GPL) permits anyone to use the code, but requires that any derived/modified source code be public
		\begin{itemize}
		\item ``Contagious'' license
		\end{itemize}
	\item ``Free software'' was/is the way GNU describes it
		\begin{itemize}
		\item Free as in speech, not free as in beer
		\end{itemize}
	\item ``Open source'' term is a little gentler
		\begin{itemize}
		\item Not interpreted by commercial interests as ``no profit''
		\item GNU also tends to have a stricter view on enforcing this ``freedom''
		\item Other licenses fall in this category (MIT, Apache, BSD, LGPL, etc)
		\end{itemize}
	\end{itemize}
\end{frame}

\begin{frame}{Who develops Linux?}
	\begin{itemize}
	\item Thousands of people around the world (1918 in Linux v6.10)
	\item Hundreds of companies (203 in Linux v6.10)
	\item Developers, reviewers, testers, maintainers, bug reporters
		\begin{itemize}
		\item \url{https://lwn.net/Articles/981559/}
		\end{itemize}
	\item Anyone \emph{can} contribute
		\begin{itemize}
		\item But your contribution will be ignored (or worse) if you do it wrong
		\end{itemize}
	\end{itemize}
\end{frame}

\begin{frame}{How is the Linux kernel source code managed?}
	\begin{itemize}
	\item Linus Torvalds ``owns'' the top-level, ``official'' source code
		\begin{itemize}
		\item 9-10 week release cycle, starting with ``-rc1'', ``-rc2'', and so on
		\item Linux v6.10 was the most recent release; v6.11 is under development
		\end{itemize}
	\item Source tree is maintained as a hierarchy of components
		\begin{itemize}
		\item Core kernel, drivers, networking, file systems, etc.
		\end{itemize}
	\item Components have \strong{maintainers} responsible for them
		\begin{itemize}
		\item Only a maintainer can accept a change to a component
		\end{itemize}
	\item Maintainers supply their accepted changes to their ``parent'' in the hierarchy
		\begin{itemize}
		\item Linux is the top-level maintainer
		\item Maintainers trust (but verify) their ``subordinates''
		\end{itemize}
	\end{itemize}
\end{frame}

\begin{frame}{Linux maintainer hierarchy}
	\includegraphics[width=25em]{maintainer_hierarchy.png}
\end{frame}

\begin{frame}{How are changes made to the kernel?}
	\begin{itemize}
	\item A developer (anyone) proposes a change
		\begin{itemize}
		\item This takes the form of a patch
		\item Shared via \strong{archived} mailing lists which are publicly available
		\item Often a ``series'' of patches is required to implement a change
		\end{itemize}
	\item The change is reviewed (by anyone, but especially the maintainer)
		\begin{itemize}
		\item Typically, improvements are suggested
		\end{itemize}
	\item New versions of the proposed change are sent, and the process repeats
	\item Eventually the maintainer accepts the change
	\end{itemize}
\end{frame}

\begin{frame}{Can \strong{you} contribute to Linux?}
	\begin{itemize}
	\item \strong{Yes!!!}  But actually, \strong{no}, not right now
	\item The University of Minnesota is currently banned from contributing
	\item The ban stems from some incidents in 2021
		\begin{itemize}
		\item Research group submitted intentionally bad commits
		\item This violates Linux contribution policy in several ways
		\item Ultimate resolution was a ban on future submissions
		\end{itemize}
	\item We aim to improve this situation, partially with this group’s activities.
	\item \strong{It is \colorbox{yellow}{VERY IMPORTANT} that no one ignore this ban}
	\end{itemize}
\end{frame}

\begin{frame}{Resources}
	\begin{itemize}
	\item The main Linux development website
		\begin{itemize}
		\item \url{https://www.kernel.org/}
		\end{itemize}
	\item Current in-tree Linux documentation
		\begin{itemize}
		\item \url{https://docs.kernel.org/}
		\end{itemize}
	\item Linux kernel mailing list archives
		\begin{itemize}
		\item \url{https://lore.kernel.org/}
		\end{itemize}
	\item Linux Weekly News
		\begin{itemize}
		\item \url{https://www.lwn.net/}
		\end{itemize}
	\end{itemize}
\end{frame}

\begin{frame}{Wrap-Up}
	\begin{itemize}
	\item TODO
		\begin{itemize}
		\item TODO
		\end{itemize}
	\end{itemize}
	\todo{write me}
\end{frame}

\end{document}
