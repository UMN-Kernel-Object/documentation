\documentclass[aspectratio=169, notes]{beamer}
% \setbeamertemplate{note page}[plain]
% \setbeameroption{show notes on second screen=right}
\usepackage{amsfonts}
\usepackage{pifont}
\usepackage{newpxmath}
\usepackage{newpxtext}

\definecolor{c1}{RGB}{ 86, 180, 233}
\definecolor{c2}{RGB}{  0, 158, 115}
\definecolor{c3}{RGB}{230, 159,   0}
\definecolor{c4}{RGB}{213,  94,   0}
\definecolor{c5}{RGB}{204, 121, 167}
\definecolor{c6}{RGB}{255, 170,  20}

\newcommand{\todo}[1]{\note[item]{\textcolor{c4}{\textbf{TODO:}} #1}}

\newcommand{\yes}{{\color{c2}\ding{51}}}
\newcommand{\no}{{\color{c4}\ding{55}}}

\title{Introduction to Lifetimes}
\author{Nathan Ringo (\texttt{remexre})}
\date{November 8, 2023}

\begin{document}

\frame{\titlepage}

\begin{frame}{Recap and outline}
	\begin{itemize}
		\item \texttt{search}
		\begin{itemize}
			\item "Imperative way"
			\item "Functional way"
		\end{itemize}
		\item References and Lifetimes
		\item Interior Mutability
	\end{itemize}
\end{frame}

\begin{frame}{\texttt{search}}
	\begin{itemize}
	\item<+-> Base: \url{https://play.rust-lang.org/?version=stable&mode=debug&edition=2021&gist=367a4490e732f842e45800ba60249795}
	\vspace{0.3in}
	\item<+-> \texttt{pub fn \href{https://doc.rust-lang.org/std/vec/struct.Vec.html\#method.retain}{\texttt{retain}}<F>(\&mut self, f: F)} \\
		\quad\texttt{where F: FnMut(\&T) -> bool} \\
		\url{https://play.rust-lang.org/?version=stable&mode=debug&edition=2021&gist=79cd0ae6e3dcc22c37242e6b0bf0bdff}
	\vspace{0.3in}
	\item<+-> Iterators and \texttt{filter}! \\
		\url{https://play.rust-lang.org/?version=stable&mode=debug&edition=2021&gist=fd52e11181bfb94d7f685ef568b72a62}
	\end{itemize}
\end{frame}

\begin{frame}{How do I access values?}
	\begin{table}
		\begin{tabular}{l|c|c|c|c}
			Type                                                  \onslide<2->{ & Read}\onslide<3->{ & Write}\onslide<4->{ & Free}\onslide<5->{ & How many?} \\
			\hline\hline
			\texttt{\textcolor{c3}{\&}\textcolor{c1}{String}}     \onslide<2->{ & \yes}\onslide<3->{ & \no  }\onslide<4->{ & \no }\onslide<5->{ & $\infty$ (but no \textcolor{c3}{\&mut })} \\
			\texttt{\textcolor{c3}{\&mut }\textcolor{c1}{String}} \onslide<2->{ & \yes}\onslide<3->{ & \yes }\onslide<4->{ & \no }\onslide<5->{ & 1} \\
			\texttt{\textcolor{c1}{String}}                       \onslide<2->{ & \yes}\onslide<3->{ & \yes }\onslide<4->{ & \yes}\onslide<5->{ & ...}
		\end{tabular}
	\end{table}

	\vspace{0.3in}

	\onslide<2->{\texttt{\textcolor{c3}{\&}}: ``shared reference'' or ``immutable reference'' (slight misnomer)}
		\begin{itemize}
		\item<2-> ``kind of like C's \texttt{const*}'' but...
		\end{itemize}

	\onslide<3->{\texttt{\textcolor{c3}{\&mut}}: ``unique reference'' or ``mutable reference'' (slight misnomer)}
		\begin{itemize}
		\item<3-> ``kind of like C's \texttt{*}'' but...
		\end{itemize}

	\vspace{0.3in}

	\onslide<2->{\texttt{pub fn len(\textcolor{c3}{\&self}) -> usize \{ ... \}}}

	\onslide<3->{\texttt{pub fn push\_str(\textcolor{c3}{\&mut self}, string: \textcolor{c3}{\&}str) \{ ... \}}}

	\onslide<4->{\texttt{pub fn into\_bytes(\textcolor{c3}{self}) -> Vec<u8> \{ ... \}}}
\end{frame}

\begin{frame}{References have lifetimes}
	\begin{itemize}
	\item \texttt{pub fn len(\textcolor{c3}{\&self}) -> usize} is implicitly \\
	\vspace{0.1in}
		\texttt{pub fn len<\textcolor{c2}{'a}>(\textcolor{c3}{\&\textcolor{c2}{'a} self}) -> usize}
	\vspace{0.4in}
	\item \texttt{pub fn push\_str(\textcolor{c3}{\&mut self}, s:\textcolor{c3}{\&}str)} is implicitly \\
	\vspace{0.1in}
		\texttt{pub fn push\_str<\textcolor{c2}{'a}>(\textcolor{c3}{\&\textcolor{c2}{'a} mut self}, s:\textcolor{c3}{\&}\textcolor{c2}{'a} str)}
	\vspace{0.4in}
	\item \texttt{fn f(\textcolor{c3}{\&mut self}, x:\textcolor{c3}{\&}T) -> \textcolor{c3}{\&}U} is implicitly \\
	\vspace{0.1in}
		\texttt{fn f<\textcolor{c2}{'a}>(\textcolor{c3}{\&\textcolor{c2}{'a} mut self}, x:\textcolor{c3}{\&}\textcolor{c2}{'a} T) -> \textcolor{c3}{\&}\textcolor{c2}{'a} U}
	\end{itemize}
\end{frame}

\begin{frame}{What are lifetimes?}
	\begin{itemize}
		\item<+-> The range of the program where a reference is valid
			\begin{itemize}
				\item If you want to make a \textcolor{c3}{\&mut} for a value, all its \textcolor{c3}{\&}s must be dead
				\item If you want to own a value, all its \textcolor{c3}{\&}s and \textcolor{c3}{\&mut}s must be dead
			\end{itemize}
		\vspace{0.2in}
		\item<+-> Since you need to own something to free it, if you have a reference (that's alive) to it, it can't have been freed!
			\begin{itemize}
				\item No use-after-free bugs!
				\item \url{https://play.rust-lang.org/?version=stable&mode=debug&edition=2021&gist=37e5d4df6251aac80838b2753b295337}
			\end{itemize}
		\vspace{0.2in}
		\item<+-> If you have a reference to a value, nobody else can have a \textcolor{c3}{\&mut} to it. For \emph{most} types, this means nobody can mutate it out from under you!
			\begin{itemize}
				\item Counterexamples: \texttt{Mutex}, \texttt{AtomicU32}, \texttt{Cell}, ...
			\end{itemize}
	\end{itemize}
\end{frame}

\begin{frame}{Interior Mutability}
	\begin{itemize}
	\item Some types let you mutate them with only a \textcolor{c3}{\&}
		\begin{itemize}
		\item This is why \textcolor{c3}{\&mut} = ``mutable reference'' is a slight misnomer
		\end{itemize}
	\vspace{0.2in}
	\item Simple example: \href{https://doc.rust-lang.org/std/cell/struct.Cell.html}{\texttt{Cell}}
	\vspace{0.2in}
	\item Example you'll use: \href{https://doc.rust-lang.org/std/sync/struct.Mutex.html}{\texttt{Mutex}}
	\vspace{0.2in}
	\item \emph{Not} bad, but ``don't do it unless you need to''
		\begin{itemize}
			\item You don't know ``nobody can mutate it out from under you''
		\end{itemize}
	\end{itemize}
\end{frame}

\end{document}
